\documentclass{article}
\begin{document}

\title{A REPORT ON FILM MAKING}
\author{Nangwale Joshua}
\maketitle

Movies are an integral part of modern day entertainment and a common pass time and leisure activity for most people. Film production is the process of making a film and involves a number of discrete stages before the finished product can be presented before audiences.
The main stages of film production are:
\begin{enumerate}
\item Development
\item Pre-production
\item Production
\item Post production
\item Distribution
\end{enumerate}

\section{Development}
This is the period in the film’s conception where the foundational elements are assembled. It involves several aspects.

\subsection{Story development}
The basic structure of the movie is developed and the screen writers come up with a scene by scene outline of the movie.

\subsection{Writing the screenplay}
Next, a screenwriter writes a screenplay over a period of several months. The screenwriter may rewrite it several times to improve several aspects of the movie.

\subsection{Financing}
The producer and screenwriter then prepare a film pitch and present it to potential financiers.They also pitch the idea to actors and directors to convince them to work on the project.

\section{Pre-production}
In pre-production, every step of actually creating the film is carefully designed and planned. The production company is created and a production office established. The film is pre-visualized by the director.

\subsection{Casting}
Casting is done by the casting directors whose job is to find actors who match the director’s specifications.

\subsection{Locations}
Locations in which the movie is to be filmed are found by the location scouts.

\subsection{Shotlist}
This is a numbered list of shots with a description of framing and other details.

\subsection{Script breakdown}
The script breakdown is the process in which every single item needed for the movie’s shoot is identified. This includes locations, props, and effects.

\subsection{Production design}
The production designer designs and oversees the production of set pieces, and arranges the procurement of anything that needs to be purchased: plants, furniture.

\section{Production}
In production the movie is actually created and shot. Additional crew are hired at this stage.

\subsection{Setting  up the shot}
The crew must report to the location at the call time and the 1st Assistant Director immediately begins to oversee the crew, and the director need not be around at this stage.

\subsection{ Rehearsal}
While the crew prepare their equipment, the actors are ward robed in their costumes and attend the hair and make-up departments. The actors rehearse the script with the director, and the camera and sound crews rehearse with them.

\subsection{Filming}
Having chosen the focal length, camera placement, the actor’s marks and other details such as camera movement, the director tells the cinematographer where to put the camera, which lens to use and the details of any camera movement.

\subsection{Checking the take}
After a take, the director reviews the take on the video monitor and decides what needs to be tweaked. The process is repeated until the director is satisfied.

\section{Post production}
This occurs after the filming of the movie is done. It involves film editing, sound mixing, adding music and adding visual effects with CGI. Test screenings may also be done with small audiences to gauge the film’s reception.

\section{Distribution}
This is the final stage in the process in which the movie is released to cinemas for viewing or on occasion directly to consumer video such as DVD or Blu-ray. Posters, and other advertising materials are published, and the film is advertised and promoted.  Most films are also promoted with their own special website.




\end{document}